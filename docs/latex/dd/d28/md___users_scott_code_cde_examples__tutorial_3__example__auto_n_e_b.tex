\subsection*{$\ast$$\ast$\+Introduction$\ast$$\ast$}

Auto\+N\+EB is an alternative to the typical N\+EB algorithm. You can find useful references to the Auto\+Neb method in the {\itshape References} (\#\+References) section of the documentation.

This example is similar to {\itshape Tutorial 2}, with the main distinction being the fact that Auto\+N\+EB is used instead of standard C\+I\+N\+EB. In Auto\+Neb, after an initial N\+EB refinement, one then adds a series of beads into the current reaction-\/string, followed by further localized refinement. This approach iteratively improves the description of the M\+EP, and often leads to smoother reaction profiles.

Because this tutorial is very similar to {\itshape Tutorial 1}, we refer to that page for further details about L\+A\+M\+M\+PS and Reax\+FF. Here, we simply highlight differences in the input file, which is as follows\+: \begin{DoxyVerb}# This input file demonstrates a AUTONEB refinement.
# We use the Quickmin optimization method here, and lammps plus Reaxff for PES calculations.
calctype optpath
# We start with 7 images with which to perform regular NEB
nimage 7
# We shall add a futher 23 images using AUTONEB, making a total of 30 at the end of the procedure
autoneb 23

# provide initial and final images in files called start and end:
startfrompath  .false.
startfile start.xyz
endfile end.xyz

# typical path optimization parameters
pathoptmethod cineb
nebmethod     quickmin
nebiter       500
nebspring     0.025
nebstep       20.0
neboutfreq    10
ngdsrelax     10000

# threashold paramters when using LAMPPS/REAXFF:
cithresh      1d-2
nebconv       8d-3
vsthresh      5d-2
# The use of these forces are recommended by the AUTONEB article
projforcetype 3

# reconnect linearly interpolates - idppguess then improves that interpolation
reconnect      .true.
idppguess      .true.
# almost always the case:
stripinactive  .true.
# For most problems, optimise before starting:
optendsbefore  .true.
optendsduring  .false.

# no restraining
nebrestrend    .false.
shimmybeads    .true.
fraginterpol   .true.
reactivegather .true.

# constraints
dofconstraints 0

# the platinum is frozen
atomconstraints 1
1

# use lammps, using lmp.min for minimizations and lmp.head for gradients and energies
pestype  lammps
pesfile   lmp.head
pesopttype  lammps
pesoptfile lmp.min
# name of the executables
pesexecutable '/Users/scott/code/lammps-22Aug18/src/lmp_serial'
pesoptexecutable '/Users/scott/code/lammps-22Aug18/src/lmp_serial'\end{DoxyVerb}
 