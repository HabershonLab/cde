The forbid-\/file is used in G\+DS to define which graph-\/patterns cannot be generated as new reaction-\/path end-\/points during each G\+DS graph-\/change step.

A typical forbid-\/file (usually called {\itshape forbid.\+in}, although can be called anything) looks like this\+: \begin{DoxyVerb}forbid
natom 3
-
0 1 0
1 0 1
0 1 0
-
labels C O O

forbid
natom 3
-
0 1 0
1 0 1
0 1 0
-
labels C O C
\end{DoxyVerb}


Note that the format is very similar to that of the moves files.

Each forbidden graph-\/patter is defined in a block, which looks like this. \begin{DoxyVerb}forbid  
natom 3
-
0 1 0
1 0 1
    0 1 0
-
labels C O O
\end{DoxyVerb}


In the example above, the bonding pattern specified by the graph (e.\+g. bond between atoms 1 and 2, bond between atoms 2 and 3) is N\+OT allowed to be generated for the atom-\/labels C,O,O. In othert words, this pattern would stop C-\/\+O-\/O, and any molecules containing this pattern, from forming.

Note that, in contrast to the move-\/file, the atom labels {\bfseries{must}} be defined in the forbid-\/file. 