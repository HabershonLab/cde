Setting up C\+DE \{\#setup\}

\subsection*{Compiling C\+DE}

Before running, C\+DE must first be compiled.

To compile\+:


\begin{DoxyItemize}
\item Go to the {\itshape src} directory.
\item Edit the {\itshape Makefile} so that your Fortran compiler and any required libraries are picked up.
\item Type {\itshape make}.
\item After compilation, you should now have an executable labelled {\itshape cde.\+x}
\end{DoxyItemize}

\subsection*{Running C\+DE}

To run the code, simply type\+: \begin{DoxyVerb}    cde.x input
\end{DoxyVerb}


where {\itshape input} is the name of your input file.

See the sections \mbox{\hyperlink{_annotated}{Annotated input file description}} and \mbox{\hyperlink{_example}{Example input file for C\+DE}} for descriptions of input files.

To make things easier, you should add an alias to your $\ast$.bashrc$\ast$ file in your home directory. To do so (assuming a linux/unix system using B\+A\+SH shell)\+:


\begin{DoxyItemize}
\item Type {\itshape cd} to change to your home directory;
\item Type {\itshape vi .bashrc}
\item Add the line
\end{DoxyItemize}

\begin{DoxyVerb}*alias cde='YYY/cde.x'*
\end{DoxyVerb}


where {\itshape Y\+YY} is the full directory path to the {\itshape cde.\+x} executable.


\begin{DoxyItemize}
\item Save the $\ast$.bashrc$\ast$ file, then type
\end{DoxyItemize}

\begin{DoxyVerb}source .bashrc
\end{DoxyVerb}



\begin{DoxyItemize}
\item If successful, you should then be able to type {\itshape cde.\+x} to run the C\+DE code. 
\end{DoxyItemize}